\subsection{main.\-c}

\begin{DoxyCode}
00001 \textcolor{comment}{/**  @file main.c}
00002 \textcolor{comment}{ *   @brief  矩形近似による数値積分}
00003 \textcolor{comment}{ *   @date   2016/12/15}
00004 \textcolor{comment}{ *   @author 佐伯雄飛}
00005 \textcolor{comment}{ *   @author B162392}
00006 \textcolor{comment}{ */}
00007 
00008 \textcolor{preprocessor}{#include <stdio.h>}
00009 \textcolor{preprocessor}{#include <stdlib.h>}  \textcolor{comment}{// atoi, atof}
00010 \textcolor{preprocessor}{#include <math.h>}    \textcolor{comment}{// sin, cos}
00011 \textcolor{preprocessor}{#include <string.h>}  \textcolor{comment}{// strncmp}
00012 \textcolor{comment}{}
00013 \textcolor{comment}{/** @fn double linear(double x)}
00014 \textcolor{comment}{ *  @brief 1次関数}
00015 \textcolor{comment}{ */}
00016 \textcolor{keywordtype}{double} linear(\textcolor{keywordtype}{double} x) \{
00017   \textcolor{keywordtype}{double} a = 2.0, b = 0.5;
00018   \textcolor{keywordflow}{return} a * x + b;
00019 \}\textcolor{comment}{}
00020 \textcolor{comment}{/** @fn double quad(double x)}
00021 \textcolor{comment}{ *  @brief 2次関数}
00022 \textcolor{comment}{ */}
00023 \textcolor{keywordtype}{double} quad(\textcolor{keywordtype}{double} x) \{
00024   \textcolor{keywordtype}{double} a = 2.0, b = 0.5, c = 1.0;
00025   \textcolor{keywordflow}{return} a * x * x + b * x + c;
00026 \}
00027 \textcolor{comment}{}
00028 \textcolor{comment}{/** @fn int main(int argc, char* argv[])}
00029 \textcolor{comment}{ *  @brief 矩形近似による数値積分}
00030 \textcolor{comment}{ *}
00031 \textcolor{comment}{ *  次の近似式による数値積分を行う.}
00032 \textcolor{comment}{ *  \(\backslash\)f$ \(\backslash\)int\_a^b f(x) dx \(\backslash\)approx \(\backslash\)sum\_\{i=1\}^\{n\} f(x\_i) \(\backslash\)frac\{b-a\}\{n\} \(\backslash\)f$}
00033 \textcolor{comment}{ *}
00034 \textcolor{comment}{ *  コマンドラインオプション:}
00035 \textcolor{comment}{ *  - 1つ目:被積分関数f(x)(linear, quad, sin, conのいずれか)}
00036 \textcolor{comment}{ *  - 2つ目:a}
00037 \textcolor{comment}{ *  - 3つ目:b,ただしa < b}
00038 \textcolor{comment}{ *  - 4つ目:n,ただし0 < n}
00039 \textcolor{comment}{ *}
00040 \textcolor{comment}{ *  出力:}
00041 \textcolor{comment}{ *  - 積分結果を小数点第5位まで表示(%.5f)}
00042 \textcolor{comment}{ *  - 次の場合は,何も表示せずに終了する(return 0で).}
00043 \textcolor{comment}{ *    被積分関数のオプションに上記4つの関数以外を指定した場合.}
00044 \textcolor{comment}{ *    aがb以上の場合.}
00045 \textcolor{comment}{ *    nが負の場合.}
00046 \textcolor{comment}{ *}
00047 \textcolor{comment}{ *  実行例:}
00048 \textcolor{comment}{\(\backslash\)verbatim}
00049 \textcolor{comment}{./main linear 0 10 1000}
00050 \textcolor{comment}{\(\backslash\)endverbatim}
00051 \textcolor{comment}{  *  出力例:}
00052 \textcolor{comment}{\(\backslash\)verbatim}
00053 \textcolor{comment}{105.10500}
00054 \textcolor{comment}{\(\backslash\)endverbatim}
00055 \textcolor{comment}{ *  実行例:}
00056 \textcolor{comment}{\(\backslash\)verbatim}
00057 \textcolor{comment}{./main sin 0 10 100000}
00058 \textcolor{comment}{\(\backslash\)endverbatim}
00059 \textcolor{comment}{  *  出力例:}
00060 \textcolor{comment}{\(\backslash\)verbatim}
00061 \textcolor{comment}{1.83904}
00062 \textcolor{comment}{\(\backslash\)endverbatim}
00063 \textcolor{comment}{ *  実行例:}
00064 \textcolor{comment}{\(\backslash\)verbatim}
00065 \textcolor{comment}{./main log 0 10 100}
00066 \textcolor{comment}{\(\backslash\)endverbatim}
00067 \textcolor{comment}{  *  出力例:}
00068 \textcolor{comment}{\(\backslash\)verbatim}
00069 \textcolor{comment}{\(\backslash\)endverbatim}
00070 \textcolor{comment}{ *  @date   2016/12/15}
00071 \textcolor{comment}{ *  @author 佐伯雄飛,B162392}
00072 \textcolor{comment}{ */}
00073 \textcolor{keywordtype}{int} main(\textcolor{keywordtype}{int} argc, \textcolor{keywordtype}{char}* argv[]) \{
00074   \textcolor{keywordflow}{if} (argc < 5) \{
00075     \textcolor{keywordflow}{return} 0;
00076   \}
00077 
00078   double (*f)(double);  \textcolor{comment}{// function pointer}
00079 
00080   \textcolor{keywordflow}{if} (strncmp(\textcolor{stringliteral}{"linear"}, argv[1], 10) == 0) \{
00081     f = linear;
00082   \} \textcolor{keywordflow}{else} \textcolor{keywordflow}{if} (strncmp(\textcolor{stringliteral}{"quad"}, argv[1], 10) == 0) \{
00083     f = quad;
00084   \} \textcolor{keywordflow}{else} \textcolor{keywordflow}{if} (strncmp(\textcolor{stringliteral}{"sin"}, argv[1], 10) == 0) \{
00085     f = sin;
00086   \} \textcolor{keywordflow}{else} \textcolor{keywordflow}{if} (strncmp(\textcolor{stringliteral}{"cos"}, argv[1], 10) == 0) \{
00087     f = cos;
00088   \} \textcolor{keywordflow}{else} \{
00089     \textcolor{keywordflow}{return} 0;
00090   \}
00091 
00092   \textcolor{keywordtype}{double} a = atof(argv[2]);
00093   \textcolor{keywordtype}{double} b = atof(argv[3]);
00094   \textcolor{keywordflow}{if} (a >= b) \{
00095     \textcolor{keywordflow}{return} 0;
00096   \}
00097   \textcolor{keywordtype}{int} n = atoi(argv[4]);
00098   \textcolor{keywordflow}{if} (n <= 0) \{
00099     \textcolor{keywordflow}{return} 0;
00100   \}
00101 
00102   \textcolor{keywordtype}{double} h = (b - a) / n;
00103   \textcolor{keywordtype}{double} integral = 0.0;
00104 
00105   \textcolor{keywordflow}{for} (\textcolor{keywordtype}{double} x = a; x <= b; x += h) \{
00106     integral += f(x) * h;
00107   \}
00108 
00109   printf(\textcolor{stringliteral}{"%.5f\(\backslash\)n"}, integral);
00110 
00111   \textcolor{keywordflow}{return} 0;
00112 \}
\end{DoxyCode}
