\subsection{main.\-c}

\begin{DoxyCode}
00001 \textcolor{comment}{/**  @file main.c}
00002 \textcolor{comment}{ *   @brief  2つの2次元整数行列の和を計算する}
00003 \textcolor{comment}{ *   @date   2016/10/18}
00004 \textcolor{comment}{ *   @author 佐伯雄飛}
00005 \textcolor{comment}{ *   @author B162392}
00006 \textcolor{comment}{ */}
00007 
00008 \textcolor{preprocessor}{#include <stdio.h>}
00009 \textcolor{comment}{}
00010 \textcolor{comment}{/** @fn int main(void)}
00011 \textcolor{comment}{ *  @brief  2つの2次元整数行列の和を計算する}
00012 \textcolor{comment}{ *}
00013 \textcolor{comment}{ *  入力:}
00014 \textcolor{comment}{ *  - 標準入力には,まず1つ目の2x2行列aの4つの要素}
00015 \textcolor{comment}{ *    (\(\backslash\)f$ a\_11,a\_12,a\_21,a\_22 \(\backslash\)f$)がこの順番で与えられる.値は整数.}
00016 \textcolor{comment}{ *  - それに引き続いて,2つ目の2x2行列bの4つの要素}
00017 \textcolor{comment}{ *    (\(\backslash\)f$ b\_11,b\_12,b\_21,b\_22 \(\backslash\)f$)がこの順番で与えられる.値は整数.}
00018 \textcolor{comment}{ *}
00019 \textcolor{comment}{ *  出力:}
00020 \textcolor{comment}{ *  - 与えられた2つの行列の和の行列cを計算し,それを標準出力に表示する.}
00021 \textcolor{comment}{ *  - 表示の1行目にはc\_11とc\_12をスペース1つで区切り表示する}
00022 \textcolor{comment}{ *  - 表示の2行目にはc\_21とc\_22をスペース1つで区切り表示する}
00023 \textcolor{comment}{ *  - それぞれの数字のprintfによる表示フォーマットは %4d とする}
00024 \textcolor{comment}{ *}
00025 \textcolor{comment}{ *  入力例:}
00026 \textcolor{comment}{\(\backslash\)verbatim}
00027 \textcolor{comment}{24 -33 -38 24 -75 -26 -46 45}
00028 \textcolor{comment}{\(\backslash\)endverbatim}
00029 \textcolor{comment}{  *  出力例:}
00030 \textcolor{comment}{\(\backslash\)verbatim}
00031 \textcolor{comment}{ -51  -59}
00032 \textcolor{comment}{ -84   69}
00033 \textcolor{comment}{\(\backslash\)endverbatim}
00034 \textcolor{comment}{  *  入力例:}
00035 \textcolor{comment}{\(\backslash\)verbatim}
00036 \textcolor{comment}{43 -7 62 26 -23 73 -53 -87}
00037 \textcolor{comment}{\(\backslash\)endverbatim}
00038 \textcolor{comment}{  *  出力例:}
00039 \textcolor{comment}{\(\backslash\)verbatim}
00040 \textcolor{comment}{  20   66}
00041 \textcolor{comment}{   9  -61}
00042 \textcolor{comment}{\(\backslash\)endverbatim}
00043 \textcolor{comment}{ *  @date   2016/10/18}
00044 \textcolor{comment}{ *  @author 佐伯雄飛,B162392}
00045 \textcolor{comment}{ */}
00046 \textcolor{keywordtype}{int} main(\textcolor{keywordtype}{void}) \{
00047   \textcolor{keywordtype}{int} i;
00048   \textcolor{keywordtype}{int} n;
00049   \textcolor{keywordtype}{int} a[i][n];
00050   \textcolor{keywordtype}{int} b[i][n];
00051   \textcolor{keywordtype}{int} c[i][n];
00052 
00053   \textcolor{keywordflow}{for} (i = 0; i <= 1; i++) \{
00054     \textcolor{keywordflow}{for} (n = 0; n <= 1; n++) \{
00055       scanf(\textcolor{stringliteral}{"%4d"}, &a[i][n]);
00056     \}
00057   \}
00058 
00059   \textcolor{keywordflow}{for} (i = 0; i <= 1; i++) \{
00060     \textcolor{keywordflow}{for} (n = 0; n <= 1; n++) \{
00061       scanf(\textcolor{stringliteral}{"%4d"}, &b[i][n]);
00062     \}
00063   \}
00064 
00065   \textcolor{keywordflow}{for} (i = 0; i <= 1; i++) \{
00066     \textcolor{keywordflow}{for} (n = 0; n <= 1; n++) \{
00067       c[i][n] = a[i][n] + b[i][n];
00068     \}
00069   \}
00070 
00071   \textcolor{keywordflow}{for} (i = 0; i <= 1; i++) \{
00072     \textcolor{keywordflow}{for} (n = 0; n <= 1; n++) \{
00073       printf(\textcolor{stringliteral}{"%4d "}, c[i][n]);
00074     \}
00075     printf(\textcolor{stringliteral}{"\(\backslash\)n"});
00076   \}
00077 
00078   \textcolor{keywordflow}{return} 0;
00079 \}
\end{DoxyCode}
