\subsection{main.\-c}

\begin{DoxyCode}
00001 \textcolor{comment}{/**  @file main.c}
00002 \textcolor{comment}{ *   @brief  2個の複素数を標準入力から読み込み,その演算結果を表示する}
00003 \textcolor{comment}{ *   @date   2016/10/27}
00004 \textcolor{comment}{ *   @author 佐伯雄飛}
00005 \textcolor{comment}{ *   @author B162392}
00006 \textcolor{comment}{ */}
00007 
00008 \textcolor{preprocessor}{#include <stdio.h>}
00009 \textcolor{comment}{}
00010 \textcolor{comment}{/** @fn int main(void)}
00011 \textcolor{comment}{ *  @brief  2個の複素数を標準入力から読み込み,その演算結果を表示する}
00012 \textcolor{comment}{ *}
00013 \textcolor{comment}{ *  入力:}
00014 \textcolor{comment}{ *  - 標準入力の最初は演算子を表す文字,+, -, *, / のいずれか.}
00015 \textcolor{comment}{ *    これ以外の文字が与えられたら,}
00016 \textcolor{comment}{\(\backslash\)verbatim}
00017 \textcolor{comment}{Error: unknown operator}
00018 \textcolor{comment}{\(\backslash\)endverbatim}
00019 \textcolor{comment}{ *    というエラーを表示し,(return 0で)終了する.}
00020 \textcolor{comment}{ *  - それに引き続く4個の実数は,演算子を適用する複素数z1とz2の実部と虚部が}
00021 \textcolor{comment}{ *    Re(z1) Im(z1) Re(z2) Im(z2) の順で与えられる}
00022 \textcolor{comment}{ *}
00023 \textcolor{comment}{ *  出力:}
00024 \textcolor{comment}{ *  - 2個の複素数に演算を適用した結果の複素数を表示する}
00025 \textcolor{comment}{ *  - 実部,虚部ともに数点第5位まで表示する(%.5f)}
00026 \textcolor{comment}{ *  - a+bjの形式で表示する.}
00027 \textcolor{comment}{ *    aは実部,bは虚部.ただしbが負の場合にはa-bjと出力する}
00028 \textcolor{comment}{ *}
00029 \textcolor{comment}{ *  入力例:}
00030 \textcolor{comment}{\(\backslash\)verbatim}
00031 \textcolor{comment}{/ 0.183514370734 0.721236372839 0.00169904140328 0.161001208126}
00032 \textcolor{comment}{\(\backslash\)endverbatim}
00033 \textcolor{comment}{  *  出力例:}
00034 \textcolor{comment}{\(\backslash\)verbatim}
00035 \textcolor{comment}{4.49122-1.09244j}
00036 \textcolor{comment}{\(\backslash\)endverbatim}
00037 \textcolor{comment}{  *  入力例:}
00038 \textcolor{comment}{\(\backslash\)verbatim}
00039 \textcolor{comment}{- 0.270864483478 0.847355883465 0.213440078194 0.417426165812}
00040 \textcolor{comment}{\(\backslash\)endverbatim}
00041 \textcolor{comment}{  *  出力例:}
00042 \textcolor{comment}{\(\backslash\)verbatim}
00043 \textcolor{comment}{0.05742+0.42993j}
00044 \textcolor{comment}{\(\backslash\)endverbatim}
00045 \textcolor{comment}{ *  @date   2016/10/27}
00046 \textcolor{comment}{ *  @author 佐伯雄飛,B162392}
00047 \textcolor{comment}{ */}
00048 \textcolor{keywordtype}{int} main(\textcolor{keywordtype}{void}) \{
00049   \textcolor{keywordtype}{char} a;
00050   \textcolor{keywordtype}{float} Re[3];
00051   \textcolor{keywordtype}{float} Im[3];
00052 
00053   scanf(\textcolor{stringliteral}{"%c"}, &a);
00054   scanf(\textcolor{stringliteral}{"%f"}, &Re[0]);
00055   scanf(\textcolor{stringliteral}{"%f"}, &Im[0]);
00056   scanf(\textcolor{stringliteral}{"%f"}, &Re[1]);
00057   scanf(\textcolor{stringliteral}{"%f"}, &Im[1]);
00058 
00059   \textcolor{keywordflow}{if} (a == \textcolor{charliteral}{'+'}) \{
00060     Re[2] = Re[0] + Re[1];
00061     Im[2] = Im[0] + Im[1];
00062   \} \textcolor{keywordflow}{else} \textcolor{keywordflow}{if} (a == \textcolor{charliteral}{'-'}) \{
00063     Re[2] = Re[0] - Re[1];
00064     Im[2] = Im[0] - Im[1];
00065   \} \textcolor{keywordflow}{else} \textcolor{keywordflow}{if} (a == \textcolor{charliteral}{'*'}) \{
00066     Re[2] = Re[0] * Re[1] - Im[0] * Im[1];
00067     Im[2] = Re[0] * Im[1] + Re[1] * Im[0];
00068   \} \textcolor{keywordflow}{else} \textcolor{keywordflow}{if} (a == \textcolor{charliteral}{'/'}) \{
00069     Re[2] = ((Re[0] * Re[1]) + (Im[0] * Im[1])) /
00070             ((Re[1] * Re[1]) + (Im[1] * Im[1]));
00071     Im[2] = ((Im[0] * Re[1]) - (Re[0] * Im[1])) /
00072             ((Re[1] * Re[1]) + (Im[1] * Im[1]));
00073   \}
00074 
00075   \textcolor{keywordflow}{if} (Im[2] > 0) \{
00076     printf(\textcolor{stringliteral}{"%.5f+%.5fj"}, Re[2], Im[2]);
00077     printf(\textcolor{stringliteral}{"\(\backslash\)n"});
00078   \}
00079   \textcolor{keywordflow}{if} (Im[2] < 0) \{
00080     printf(\textcolor{stringliteral}{"%.5f%.5fj"}, Re[2], Im[2]);
00081     printf(\textcolor{stringliteral}{"\(\backslash\)n"});
00082   \}
00083 
00084   \textcolor{keywordflow}{return} 0;
00085 \}
\end{DoxyCode}
