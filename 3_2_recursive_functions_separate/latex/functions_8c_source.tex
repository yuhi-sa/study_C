\subsection{functions.\-c}

\begin{DoxyCode}
00001 \textcolor{comment}{}
00002 \textcolor{comment}{/** @fn int Fibonacci(int n)}
00003 \textcolor{comment}{ *  @brief  nについてのフィボナッチ数を計算する}
00004 \textcolor{comment}{ *}
00005 \textcolor{comment}{ *  フィボナッチ数\(\backslash\)f$ F\_n \(\backslash\)f$は漸化式}
00006 \textcolor{comment}{ *  \(\backslash\)f$ F\_n = F\_\{n-1\} + F\_\{n-2\} \(\backslash\)f$で定義される.}
00007 \textcolor{comment}{ *  ただし\(\backslash\)f$ F\_1 = F\_2 = 1 \(\backslash\)f$.}
00008 \textcolor{comment}{ *}
00009 \textcolor{comment}{ *  @param  n 正の整数}
00010 \textcolor{comment}{ *  @return フィボナッチ数(n)}
00011 \textcolor{comment}{ *  @date   2016/11/10}
00012 \textcolor{comment}{ *  @author 佐伯雄飛,B162392}
00013 \textcolor{comment}{ */}
00014 \textcolor{keywordtype}{int} Fibonacci(\textcolor{keywordtype}{int} n) \{
00015   \textcolor{keywordtype}{int} i;
00016   \textcolor{keywordtype}{int} f0 = 0;
00017   \textcolor{keywordtype}{int} f1 = 1;
00018   \textcolor{keywordtype}{int} f2 = 1;
00019 
00020   \textcolor{keywordflow}{for} (i = 1; i < n; i++) \{
00021     f2 = f1 + f0;
00022     f0 = f1;
00023     f1 = f2;
00024   \}
00025 
00026   \textcolor{keywordflow}{return} f2;
00027 \}
00028 \textcolor{comment}{}
00029 \textcolor{comment}{/** @fn int factorial(int n)}
00030 \textcolor{comment}{ *  @brief  nの階乗を計算する}
00031 \textcolor{comment}{ *  @param  n 正の整数}
00032 \textcolor{comment}{ *  @return nの階乗}
00033 \textcolor{comment}{ *  @date   2016/11/10}
00034 \textcolor{comment}{ *  @author 佐伯雄飛,B162392}
00035 \textcolor{comment}{ */}
00036 \textcolor{keywordtype}{int} factorial(\textcolor{keywordtype}{int} n) \{
00037   \textcolor{keywordtype}{int} m;
00038   \textcolor{keywordtype}{int} f3 = 1;
00039 
00040   \textcolor{keywordflow}{for} (m = 1; m <= n; ++m) \{
00041     f3 = f3 * m;
00042   \}
00043 
00044   \textcolor{keywordflow}{return} f3;
00045 \}
\end{DoxyCode}
