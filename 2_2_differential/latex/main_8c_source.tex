\subsection{main.\-c}

\begin{DoxyCode}
00001 \textcolor{comment}{/** @file main.c}
00002 \textcolor{comment}{ *  @brief  数式微分を行う}
00003 \textcolor{comment}{ *  @date   2016/10/26}
00004 \textcolor{comment}{ *  @author 佐伯雄飛}
00005 \textcolor{comment}{ *  @author B162392}
00006 \textcolor{comment}{ */}
00007 
00008 \textcolor{preprocessor}{#include <stdio.h>}
00009 \textcolor{comment}{}
00010 \textcolor{comment}{/** @fn int main(void)}
00011 \textcolor{comment}{ *  @brief  数式微分を行う}
00012 \textcolor{comment}{ *}
00013 \textcolor{comment}{ *  標準入力からn次多項式の係数を読み込み,}
00014 \textcolor{comment}{ *  その多項式と,微分したn-1次多項式を表示する}
00015 \textcolor{comment}{ *}
00016 \textcolor{comment}{ *  入力:}
00017 \textcolor{comment}{ *  - 標準入力の最初の整数は,多項式の次数 n(nは0以上の整数)}
00018 \textcolor{comment}{ *  - それに引き続くn個の実数はn次多項式の係数}
00019 \textcolor{comment}{ *}
00020 \textcolor{comment}{ *  出力:}
00021 \textcolor{comment}{ *  - n次多項式を以下のフォーマットで出力}
00022 \textcolor{comment}{ *    c0+c1x+c2x^2+...+cnx^n}
00023 \textcolor{comment}{ *  - 微分したn-1次多項式を以下のフォーマットで出力}
00024 \textcolor{comment}{ *    c0+c1x+c2x^2+...+cnx^n-1}
00025 \textcolor{comment}{ *  - 空白は表示しない}
00026 \textcolor{comment}{ *  - 係数が負であれば,+記号を-にする}
00027 \textcolor{comment}{ *  - 定数係数(c0)が負であれば-記号を表示する.正であれば記号を表示しない}
00028 \textcolor{comment}{ *  - 係数は小数点第1位まで表示する(%.1f).}
00029 \textcolor{comment}{ *  -}
00030 \textcolor{comment}{次数nは,2次以降の項は整数だけを表示しx^nとする.1次の項はxと表示する.定数項はxを表示しない.}
00031 \textcolor{comment}{ *}
00032 \textcolor{comment}{ *  注意:}
00033 \textcolor{comment}{ *  - 次数nが負の場合には,何も表示せずに終了する(return 0で)}
00034 \textcolor{comment}{ *  - 次数nが0の場合は定数項だけであり,微分したら0を表示する}
00035 \textcolor{comment}{ *}
00036 \textcolor{comment}{ *  入力例:}
00037 \textcolor{comment}{\(\backslash\)verbatim}
00038 \textcolor{comment}{5 0.299015168361 -0.634498680199 0.076517409876 -0.155687714204 0.50453614886}
00039 \textcolor{comment}{0.260700633183}
00040 \textcolor{comment}{\(\backslash\)endverbatim}
00041 \textcolor{comment}{  *  出力例:}
00042 \textcolor{comment}{\(\backslash\)verbatim}
00043 \textcolor{comment}{0.3-0.6x+0.1x^2-0.2x^3+0.5x^4+0.3x^5}
00044 \textcolor{comment}{-0.6+0.2x-0.5x^2+2.0x^3+1.3x^4}
00045 \textcolor{comment}{\(\backslash\)endverbatim}
00046 \textcolor{comment}{  *  入力例:}
00047 \textcolor{comment}{\(\backslash\)verbatim}
00048 \textcolor{comment}{-1}
00049 \textcolor{comment}{\(\backslash\)endverbatim}
00050 \textcolor{comment}{  *  出力例:}
00051 \textcolor{comment}{\(\backslash\)verbatim}
00052 \textcolor{comment}{\(\backslash\)endverbatim}
00053 \textcolor{comment}{  *  入力例:}
00054 \textcolor{comment}{\(\backslash\)verbatim}
00055 \textcolor{comment}{0 0.338757150218}
00056 \textcolor{comment}{\(\backslash\)endverbatim}
00057 \textcolor{comment}{  *  出力例:}
00058 \textcolor{comment}{\(\backslash\)verbatim}
00059 \textcolor{comment}{0.3}
00060 \textcolor{comment}{0.0}
00061 \textcolor{comment}{\(\backslash\)endverbatim}
00062 \textcolor{comment}{ *  @date   2016/10/26}
00063 \textcolor{comment}{ *  @author 佐伯雄飛,B162392}
00064 \textcolor{comment}{ */}
00065 \textcolor{keywordtype}{int} main(\textcolor{keywordtype}{void}) \{
00066   \textcolor{keywordtype}{int} n;  \textcolor{comment}{// order of polynomial}
00067   scanf(\textcolor{stringliteral}{"%d"}, &n);
00068   n++;
00069 
00070   \textcolor{keywordflow}{if} (n < 1) \{
00071     printf(\textcolor{stringliteral}{""});
00072   \}
00073 
00074   \textcolor{keywordflow}{else} \{
00075     \textcolor{keywordtype}{float} c[n];  \textcolor{comment}{// coefficient of n-th order term}
00076     \textcolor{keywordflow}{for} (\textcolor{keywordtype}{int} i = 0; i < n; i++) \{
00077       scanf(\textcolor{stringliteral}{"%f"}, &c[i]);
00078     \}
00079 
00080     \textcolor{comment}{// print the polynomial: c0 + c1 x + c2 x^2 + .... + cn x^n}
00081 
00082     \textcolor{keywordflow}{for} (\textcolor{keywordtype}{int} i = 0; i < n; i++) \{
00083       printf(i > 1 ? \textcolor{stringliteral}{"%+.1fx^%d"} : (i == 0 ? \textcolor{stringliteral}{"%.1f"} : \textcolor{stringliteral}{"%+.1fx"}), c[i], i);
00084     \}
00085     printf(\textcolor{stringliteral}{"\(\backslash\)n"});
00086 
00087     \textcolor{keywordflow}{if} (n == 1) \{
00088       printf(\textcolor{stringliteral}{"0.0"});
00089       printf(\textcolor{stringliteral}{"\(\backslash\)n"});
00090     \}
00091 
00092     \textcolor{keywordflow}{else} \{
00093       \textcolor{comment}{// differenciation}
00094       \textcolor{keywordflow}{for} (\textcolor{keywordtype}{int} i = 1; i < n; i++) \{
00095         c[i] *= i;
00096         c[i - 1] = c[i];
00097       \}
00098 
00099       \textcolor{comment}{// print the derivative: c0' + c1' x + c2' x^2 + .... + cn' x^n-1}
00100       \textcolor{keywordflow}{for} (\textcolor{keywordtype}{int} i = 0; i < n - 1; i++) \{
00101         printf(i > 1 ? \textcolor{stringliteral}{"%+.1fx^%d"} : (i == 0 ? \textcolor{stringliteral}{"%.1f"} : \textcolor{stringliteral}{"%+.1fx"}), c[i], i);
00102       \}
00103       printf(\textcolor{stringliteral}{"\(\backslash\)n"});
00104     \}
00105   \}
00106   \textcolor{keywordflow}{return} 0;
00107 \}
\end{DoxyCode}
