\subsection{main.\-c}

\begin{DoxyCode}
00001 \textcolor{comment}{/**  @file main.c}
00002 \textcolor{comment}{ *   @brief  ベクトルxの要素をノルムの2乗で割ったものを表示する}
00003 \textcolor{comment}{ *   @date   2016/10/17}
00004 \textcolor{comment}{ *   @author 佐伯雄飛}
00005 \textcolor{comment}{ *   @author B162392}
00006 \textcolor{comment}{ */}
00007 
00008 \textcolor{preprocessor}{#include <stdio.h>}
00009 \textcolor{comment}{}
00010 \textcolor{comment}{/** @fn int main(void)}
00011 \textcolor{comment}{ *  @brief  n次元ベクトルxを読み込み,ノルムの2乗で割る}
00012 \textcolor{comment}{ *}
00013 \textcolor{comment}{ *  入力:}
00014 \textcolor{comment}{ *  - 標準入力の最初の整数はn(nは1以上の整数)}
00015 \textcolor{comment}{ *  - それに引き続くn個の実数はベクトルxの要素}
00016 \textcolor{comment}{ *    (\(\backslash\)f$x = (x\_1, x\_2, \(\backslash\)ldots, x\_n) \(\backslash\)f$)}
00017 \textcolor{comment}{ *}
00018 \textcolor{comment}{ *  出力:}
00019 \textcolor{comment}{ *  - ベクトルxの各要素\(\backslash\)f$x\_i\(\backslash\)f$を}
00020 \textcolor{comment}{ *    xのノルムの2乗で割ったもの(\(\backslash\)f$ \(\backslash\)frac\{x\_i\}\{\(\backslash\)| x \(\backslash\)|^2\} \(\backslash\)f$)を,}
00021 \textcolor{comment}{ *    縦1列に小数点第6位まで標準出力に表示する(%.6f).}
00022 \textcolor{comment}{ *  - ただしxのノルムが0の場合には,0ベクトルを表示すること}
00023 \textcolor{comment}{ *}
00024 \textcolor{comment}{ *  入力例:}
00025 \textcolor{comment}{\(\backslash\)verbatim}
00026 \textcolor{comment}{3 0.7995454735237852 0.6355523363250306 0.4970340257100355}
00027 \textcolor{comment}{\(\backslash\)endverbatim}
00028 \textcolor{comment}{  *  出力例:}
00029 \textcolor{comment}{\(\backslash\)verbatim}
00030 \textcolor{comment}{0.619686}
00031 \textcolor{comment}{0.492584}
00032 \textcolor{comment}{0.385225}
00033 \textcolor{comment}{\(\backslash\)endverbatim}
00034 \textcolor{comment}{  *  入力例:}
00035 \textcolor{comment}{\(\backslash\)verbatim}
00036 \textcolor{comment}{8 0.192413183882287 0.3592929421318327 0.43304124430348645 0.04812775140485881}
00037 \textcolor{comment}{0.8097877087349868 0.4454853632738086 0.5002661497859434 0.2942751129005189}
00038 \textcolor{comment}{\(\backslash\)endverbatim}
00039 \textcolor{comment}{  *  出力例:}
00040 \textcolor{comment}{\(\backslash\)verbatim}
00041 \textcolor{comment}{0.124376}
00042 \textcolor{comment}{0.232247}
00043 \textcolor{comment}{0.279917}
00044 \textcolor{comment}{0.031110}
00045 \textcolor{comment}{0.523446}
00046 \textcolor{comment}{0.287961}
00047 \textcolor{comment}{0.323371}
00048 \textcolor{comment}{0.190219}
00049 \textcolor{comment}{\(\backslash\)endverbatim}
00050 \textcolor{comment}{ *  @date   2016/10/17}
00051 \textcolor{comment}{ *  @author 佐伯雄飛,B162392}
00052 \textcolor{comment}{ */}
00053 \textcolor{keywordtype}{int} main(\textcolor{keywordtype}{void}) \{
00054   \textcolor{keywordtype}{int} length;
00055   scanf(\textcolor{stringliteral}{"%d"}, &length);
00056 
00057   \textcolor{keywordtype}{float} x[length];
00058   \textcolor{keywordtype}{float} sum\_x = 0.0;
00059   \textcolor{keywordflow}{for} (\textcolor{keywordtype}{int} i = 0; i < length; i++) \{
00060     scanf(\textcolor{stringliteral}{"%f"},
00061           &x[i]);  \textcolor{comment}{// ここで標準入力から実数を読み込み,x[i]に代入してください}
00062 
00063     sum\_x += x[i] * x[i];
00064   \}
00065 
00066   \textcolor{keywordflow}{for} (\textcolor{keywordtype}{int} i = 0; i < length; i++) \{
00067     \textcolor{keywordflow}{if} (sum\_x > 0) \{
00068       x[i] /= sum\_x;
00069     \} \textcolor{keywordflow}{else} \{
00070       x[i] = 0.0;
00071     \}
00072 
00073     printf(\textcolor{stringliteral}{"%.6f\(\backslash\)n"}, x[i]);  \textcolor{comment}{// ここでx[i]を表示してください}
00074   \}
00075 
00076   \textcolor{keywordflow}{return} 0;
00077 \}
\end{DoxyCode}
