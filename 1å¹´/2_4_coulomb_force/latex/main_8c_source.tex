\subsection{main.\-c}

\begin{DoxyCode}
00001 \textcolor{comment}{/**  @file main.c}
00002 \textcolor{comment}{ *   @brief  2つの静電荷間に働く力の計算}
00003 \textcolor{comment}{ *   @date   2016/10/27}
00004 \textcolor{comment}{ *   @author 佐伯雄飛}
00005 \textcolor{comment}{ *   @author B162392}
00006 \textcolor{comment}{ */}
00007 
00008 \textcolor{preprocessor}{#include <stdio.h>}
00009 \textcolor{comment}{}
00010 \textcolor{comment}{/** @fn int main(void)}
00011 \textcolor{comment}{ *  @brief  2つの静電荷間に働く力の計算}
00012 \textcolor{comment}{ *}
00013 \textcolor{comment}{ *  標準入力から2つの静電荷の電荷量と3次元座標を読み込み,}
00014 \textcolor{comment}{ *  その間に働く力をクーロンの法則により求める}
00015 \textcolor{comment}{ *}
00016 \textcolor{comment}{ *  入力:}
00017 \textcolor{comment}{ *  - 標準入力の最初の実数は,1つ目の静電荷の電荷量}
00018 \textcolor{comment}{ *  - それに引き続く3個の実数は1つ目の静電荷の座標(x,y,z)}
00019 \textcolor{comment}{ *  - 次に続く実数は,2つ目の静電荷の電荷量}
00020 \textcolor{comment}{ *  - それに引き続く3個の実数は2つ目の静電荷の座標(x,y,z)}
00021 \textcolor{comment}{ *}
00022 \textcolor{comment}{ *  出力:}
00023 \textcolor{comment}{ *  - 2つの静電荷間に働く力を計算し,表示する.}
00024 \textcolor{comment}{ *  - 数値は指数形式で,小数点第6位まで表示する(%.6e).}
00025 \textcolor{comment}{ *}
00026 \textcolor{comment}{ *  入力例:}
00027 \textcolor{comment}{\(\backslash\)verbatim}
00028 \textcolor{comment}{0.000308699789318}
00029 \textcolor{comment}{0.000402224300987 0.000279583777408 0.000174189877294}
00030 \textcolor{comment}{-0.000239153024041}
00031 \textcolor{comment}{-5.80356914685e-05 -0.000160431721634 -6.19990827336e-05}
00032 \textcolor{comment}{\(\backslash\)endverbatim}
00033 \textcolor{comment}{  *  出力例:}
00034 \textcolor{comment}{\(\backslash\)verbatim}
00035 \textcolor{comment}{1.440554e+09}
00036 \textcolor{comment}{\(\backslash\)endverbatim}
00037 \textcolor{comment}{  *  入力例:}
00038 \textcolor{comment}{\(\backslash\)verbatim}
00039 \textcolor{comment}{5.89869864357e-05}
00040 \textcolor{comment}{-0.000307976825127 -0.000255850688569 -0.000440580236465}
00041 \textcolor{comment}{-0.000381973244918}
00042 \textcolor{comment}{-0.000129578394884 -0.000480165202626 -0.000181748151379}
00043 \textcolor{comment}{\(\backslash\)endverbatim}
00044 \textcolor{comment}{  *  出力例:}
00045 \textcolor{comment}{\(\backslash\)verbatim}
00046 \textcolor{comment}{-1.359709e+09}
00047 \textcolor{comment}{\(\backslash\)endverbatim}
00048 \textcolor{comment}{ *}
00049 \textcolor{comment}{ *  @date   2016/10/27}
00050 \textcolor{comment}{ *  @author 佐伯雄飛,B162392}
00051 \textcolor{comment}{ */}
00052 
00053 \textcolor{keywordtype}{int} main(\textcolor{keywordtype}{void}) \{
00054   \textcolor{keywordtype}{float} k0 = 9.0e9;      \textcolor{comment}{// about 9.0 x 10^9 [N m^2 / C^2]}
00055   \textcolor{keywordtype}{float} q[2];            \textcolor{comment}{// charges of two particles [C]}
00056   \textcolor{keywordtype}{float} r[2][3];         \textcolor{comment}{// 3D coordinates of two particles [m]}
00057   \textcolor{keywordtype}{float} distance = 0.0;  \textcolor{comment}{// (square of) distance between two particles}
00058   \textcolor{keywordtype}{float} tempt;
00059   \textcolor{keywordflow}{for} (\textcolor{keywordtype}{int} i = 0; i < 2; i++) \{
00060     scanf(\textcolor{stringliteral}{"%e"}, &q[i]);  \textcolor{comment}{// ここでqを読み込む}
00061     \textcolor{keywordflow}{for} (\textcolor{keywordtype}{int} j = 0; j < 3; j++) \{
00062       scanf(\textcolor{stringliteral}{"%e"}, &r[i][j]);  \textcolor{comment}{// ここでrを読み込む}
00063     \}
00064   \}
00065 
00066   \textcolor{keywordflow}{for} (\textcolor{keywordtype}{int} j = 0; j < 3; j++) \{
00067     \textcolor{keywordtype}{float} temp = r[0][j] - r[1][j];
00068     distance += temp * temp;
00069   \}
00070   \textcolor{comment}{// ここでdistanceを計算する}
00071 
00072   \textcolor{keywordtype}{float} F;
00073   \textcolor{keywordflow}{if} (distance > 0) \{
00074     F = k0 * q[0] * q[1] / distance;  \textcolor{comment}{// [N]}
00075   \} \textcolor{keywordflow}{else} \{
00076     F = 0;
00077   \}
00078 
00079   printf(\textcolor{stringliteral}{"%.6e"}, F);
00080   printf(\textcolor{stringliteral}{"\(\backslash\)n"});
00081   \textcolor{comment}{// ここでFを表示する}
00082 
00083   \textcolor{keywordflow}{return} 0;
00084 \}
\end{DoxyCode}
