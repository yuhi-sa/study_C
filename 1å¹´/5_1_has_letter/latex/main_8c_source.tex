\subsection{main.\-c}

\begin{DoxyCode}
00001 \textcolor{comment}{/**  @file main.c}
00002 \textcolor{comment}{ *   @brief  文字列中の文字の有無を判定する}
00003 \textcolor{comment}{ *   @date   2016/12/15}
00004 \textcolor{comment}{ *   @author 佐伯雄飛}
00005 \textcolor{comment}{ *   @author B162392}
00006 \textcolor{comment}{ */}
00007 
00008 \textcolor{preprocessor}{#include <stdio.h>}
00009 \textcolor{preprocessor}{#include <stdbool.h>}
00010 \textcolor{comment}{}
00011 \textcolor{comment}{/** @fn bool find\_letter(char x, char str[])}
00012 \textcolor{comment}{ *  @brief  文字列str中の文字xの位置を返す}
00013 \textcolor{comment}{ *}
00014 \textcolor{comment}{ *  仕様:}
00015 \textcolor{comment}{ *  - 文字列中にxが存在すればtrue,なければfalse}
00016 \textcolor{comment}{ *}
00017 \textcolor{comment}{ *  @param  x 検索する文字(char型)}
00018 \textcolor{comment}{ *  @param  str 検索対象の文字列(char[]型)}
00019 \textcolor{comment}{ *  @return bool型.str中にxが存在すればtrue,そうでなければfalse}
00020 \textcolor{comment}{ *  @date   2016/12/15}
00021 \textcolor{comment}{ *  @author 佐伯雄飛,B162392}
00022 \textcolor{comment}{ */}
00023 
00024 \textcolor{keywordtype}{bool} find_letter(\textcolor{keywordtype}{char} x, \textcolor{keywordtype}{char} str[]) \{
00025   \textcolor{keywordtype}{int} i = 0;
00026   \textcolor{keywordflow}{while} (str[i] != \textcolor{charliteral}{'\(\backslash\)0'}) \{
00027     \textcolor{keywordflow}{if} (x == str[i]) \{
00028       \textcolor{keywordflow}{return} \textcolor{keyword}{true};
00029     \}
00030     i++;
00031   \}
00032   \textcolor{keywordflow}{return} \textcolor{keyword}{false};
00033 \}
00034 \textcolor{comment}{}
00035 \textcolor{comment}{/** @fn int main(void)}
00036 \textcolor{comment}{ *  @brief 文字列中の文字の有無を判定する}
00037 \textcolor{comment}{ *}
00038 \textcolor{comment}{ *  入力:}
00039 \textcolor{comment}{ *  - コマンドライン引数に検索するべき文字xが1つ与えられる}
00040 \textcolor{comment}{ *  -}
00041 \textcolor{comment}{標準入力には検索対象の,空白を含まない文字列strが1つ与えられる.長さはlen以下.}
00042 \textcolor{comment}{ *}
00043 \textcolor{comment}{ *  出力:}
00044 \textcolor{comment}{ *  - 文字列strに文字xが存在すれば1を,そうでなければ0を表示する.}
00045 \textcolor{comment}{ *  - コマンドライン引数にxが与えられなければ,何も表示せずに終了する(return}
00046 \textcolor{comment}{0で).}
00047 \textcolor{comment}{ *}
00048 \textcolor{comment}{ *  実行例:}
00049 \textcolor{comment}{\(\backslash\)verbatim}
00050 \textcolor{comment}{./main G}
00051 \textcolor{comment}{\(\backslash\)endverbatim}
00052 \textcolor{comment}{ *  入力例:}
00053 \textcolor{comment}{\(\backslash\)verbatim}
00054 \textcolor{comment}{Hqb5GF&M3ilq8HlAyhz7aL9OYjIbZXI1A}
00055 \textcolor{comment}{\(\backslash\)endverbatim}
00056 \textcolor{comment}{  *  出力例:}
00057 \textcolor{comment}{\(\backslash\)verbatim}
00058 \textcolor{comment}{1}
00059 \textcolor{comment}{\(\backslash\)endverbatim}
00060 \textcolor{comment}{ *  実行例:}
00061 \textcolor{comment}{\(\backslash\)verbatim}
00062 \textcolor{comment}{./main X}
00063 \textcolor{comment}{\(\backslash\)endverbatim}
00064 \textcolor{comment}{  *  入力例:}
00065 \textcolor{comment}{\(\backslash\)verbatim}
00066 \textcolor{comment}{JwaH}
00067 \textcolor{comment}{\(\backslash\)endverbatim}
00068 \textcolor{comment}{  *  出力例:}
00069 \textcolor{comment}{\(\backslash\)verbatim}
00070 \textcolor{comment}{0}
00071 \textcolor{comment}{\(\backslash\)endverbatim}
00072 \textcolor{comment}{ *  実行例:}
00073 \textcolor{comment}{\(\backslash\)verbatim}
00074 \textcolor{comment}{./main}
00075 \textcolor{comment}{\(\backslash\)endverbatim}
00076 \textcolor{comment}{  *  入力例:}
00077 \textcolor{comment}{\(\backslash\)verbatim}
00078 \textcolor{comment}{asflk;j;lfas}
00079 \textcolor{comment}{\(\backslash\)endverbatim}
00080 \textcolor{comment}{  *  出力例:}
00081 \textcolor{comment}{\(\backslash\)verbatim}
00082 \textcolor{comment}{\(\backslash\)endverbatim}
00083 \textcolor{comment}{ *  @date   2016/12/15}
00084 \textcolor{comment}{ *  @author 佐伯雄飛,B162392}
00085 \textcolor{comment}{ */}
00086 \textcolor{keywordtype}{int} main(\textcolor{keywordtype}{int} argc, \textcolor{keywordtype}{char}* argv[]) \{
00087   \textcolor{comment}{// 引数の数を判定してください}
00088   \textcolor{keywordtype}{char} x = *argv[1];  \textcolor{comment}{// コマンドライン引数からxに値を設定してください}
00089   \textcolor{keywordtype}{int} len = 100;
00090 
00091   \textcolor{keywordtype}{char} str[len];
00092   scanf(\textcolor{stringliteral}{"%99s"}, str);
00093   \textcolor{keywordtype}{bool} location = find_letter(x, str);
00094 
00095   printf(\textcolor{stringliteral}{"%d\(\backslash\)n"}, location ? 1 : 0);
00096 
00097   \textcolor{keywordflow}{return} 0;
00098 \}
\end{DoxyCode}
