\subsection{main.\-c}

\begin{DoxyCode}
00001 \textcolor{comment}{/**  @file main.c}
00002 \textcolor{comment}{ *   @brief  文字列中の,大文字と小文字のヒストグラムを作成する}
00003 \textcolor{comment}{ *   @date   2016/12/15}
00004 \textcolor{comment}{ *   @author 佐伯雄飛}
00005 \textcolor{comment}{ *   @author B162392}
00006 \textcolor{comment}{ */}
00007 
00008 \textcolor{preprocessor}{#include <stdio.h>}
00009 \textcolor{preprocessor}{#include <stdlib.h>}  \textcolor{comment}{// malloc}
00010 \textcolor{preprocessor}{#include <ctype.h>}   \textcolor{comment}{// isuppter, islower}
00011 \textcolor{comment}{}
00012 \textcolor{comment}{/** @fn int main(int argc, char* argv[])}
00013 \textcolor{comment}{ *  @brief 文字列中の,大文字と小文字のヒストグラムを作成する}
00014 \textcolor{comment}{ *}
00015 \textcolor{comment}{ *  入力:}
00016 \textcolor{comment}{ *  - 標準入力には検索対象の,空白と改行を含む文字列が与えられる.}
00017 \textcolor{comment}{ *    長さは不明(ただしmax\_int以下とする).}
00018 \textcolor{comment}{ *}
00019 \textcolor{comment}{ *  出力:}
00020 \textcolor{comment}{ *  - 文字列に含まれる大文字の数を,AからZの順番に(%5dで)表示し,改行する.}
00021 \textcolor{comment}{ *  - 文字列に含まれる小文字の数を,aからzの順番に(%5dで)表示し,改行する.}
00022 \textcolor{comment}{ *}
00023 \textcolor{comment}{ *  入力例:}
00024 \textcolor{comment}{\(\backslash\)verbatim}
00025 \textcolor{comment}{dpMhhyPmc yI DT iJUXpNfh RpV PPWJGvcE nuUB jqYjBrPqs}
00026 \textcolor{comment}{cS ysO rDE eRiZNQS YjwrGKI Ujg}
00027 \textcolor{comment}{OeurKNK zTSGVHej SxR iDmueNb}
00028 \textcolor{comment}{\(\backslash\)endverbatim}
00029 \textcolor{comment}{  *  出力例(フォーマットの都合上改行されています):}
00030 \textcolor{comment}{\(\backslash\)verbatim}
00031 \textcolor{comment}{    0     2     0     3     2     0     3     1     2     2     3     0     1}
00032 \textcolor{comment}{4     2     4     1     3     4     2     3     2     1     1     2     1}
00033 \textcolor{comment}{    0     1     3     1     4     1     1     3     3     5     0     0     2}
00034 \textcolor{comment}{1     0     3     2     4     2     0     3     1     1     1     3     1}
00035 \textcolor{comment}{\(\backslash\)endverbatim}
00036 \textcolor{comment}{  *  入力例:}
00037 \textcolor{comment}{\(\backslash\)verbatim}
00038 \textcolor{comment}{YjbiSz VdGaIJ}
00039 \textcolor{comment}{kJSJQNxNX YoPZY Rwe}
00040 \textcolor{comment}{FX zUUCazG lttNJ dVeNwI ZMEyXkOmg XGHW VBHJ}
00041 \textcolor{comment}{MkcR LdZT KHO LI InoJhFI}
00042 \textcolor{comment}{Yo JKSBedv CEgYo xvulXS pIIC wFj Lw}
00043 \textcolor{comment}{MRvqx mz gMMhJ}
00044 \textcolor{comment}{qfDxEn DNX wdtGFdtBa uektgqxS EKDteMYG pJq}
00045 \textcolor{comment}{Wdi XXLjM scXkn soxPTZs yNQggfT WTpEb}
00046 \textcolor{comment}{\(\backslash\)endverbatim}
00047 \textcolor{comment}{  *  出力例(フォーマットの都合上改行されています):}
00048 \textcolor{comment}{\(\backslash\)verbatim}
00049 \textcolor{comment}{    0     3     3     3     5     4     5     3     7     9     3     4     7}
00050 \textcolor{comment}{6     2     2     2     3     5     4     2     3     3     9     6     4}
00051 \textcolor{comment}{    3     2     2     7     5     2     6     2     2     3     5     2     2}
00052 \textcolor{comment}{3     5     3     4     0     3     6     2     3     5     6     2     4}
00053 \textcolor{comment}{\(\backslash\)endverbatim}
00054 \textcolor{comment}{ *  @date   2016/12/15}
00055 \textcolor{comment}{ *  @author 佐伯雄飛,B162392}
00056 \textcolor{comment}{ */}
00057 \textcolor{keywordtype}{int} main(\textcolor{keywordtype}{int} argc, \textcolor{keywordtype}{char}* argv[]) \{
00058   \textcolor{keywordtype}{int}* lower\_letter\_histogram;
00059   \textcolor{keywordtype}{int}* upper\_letter\_histogram;
00060 
00061   lower\_letter\_histogram = (\textcolor{keywordtype}{int}*)malloc(26 * \textcolor{keyword}{sizeof}(\textcolor{keywordtype}{int}));
00062   upper\_letter\_histogram = (\textcolor{keywordtype}{int}*)malloc(26 * \textcolor{keyword}{sizeof}(\textcolor{keywordtype}{int}));
00063   \textcolor{keywordflow}{if} (lower\_letter\_histogram == NULL || upper\_letter\_histogram == NULL) \{
00064     printf(\textcolor{stringliteral}{"can't alloc memory\(\backslash\)n"});
00065     exit(0);
00066   \}
00067 
00068   \textcolor{keywordtype}{char} c;
00069   \textcolor{keywordflow}{while} ((c = getchar()) != EOF) \{  \textcolor{comment}{// EOF is the end of stdin}
00070     \textcolor{keywordflow}{if} (isupper(c)) \{
00071       upper\_letter\_histogram[c -= \textcolor{charliteral}{'A'}]++;
00072       \textcolor{comment}{// 大文字ヒストグラム用のコードを書いてください}
00073     \}
00074     \textcolor{keywordflow}{if} (islower(c)) \{
00075       lower\_letter\_histogram[c -= \textcolor{charliteral}{'a'}]++;
00076       \textcolor{comment}{// 小文字ヒストグラム用のコードを書いてください}
00077     \}
00078   \}
00079 
00080   \textcolor{keywordflow}{for} (\textcolor{keywordtype}{int} i = 0; i < 26; i++) \{
00081     printf(\textcolor{stringliteral}{"%5d "}, upper\_letter\_histogram[i]);
00082   \}
00083   printf(\textcolor{stringliteral}{"\(\backslash\)n"});
00084   \textcolor{keywordflow}{for} (\textcolor{keywordtype}{int} i = 0; i < 26; i++) \{
00085     printf(\textcolor{stringliteral}{"%5d "}, lower\_letter\_histogram[i]);
00086   \}
00087   printf(\textcolor{stringliteral}{"\(\backslash\)n"});
00088 
00089   free(lower\_letter\_histogram);
00090   free(upper\_letter\_histogram);
00091 
00092   \textcolor{keywordflow}{return} 0;
00093 \}
\end{DoxyCode}
