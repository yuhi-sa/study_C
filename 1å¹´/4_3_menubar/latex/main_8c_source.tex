\subsection{main.\-c}

\begin{DoxyCode}
00001 \textcolor{comment}{/**  @file main.c}
00002 \textcolor{comment}{ *   @brief  メニュー項目を表示する}
00003 \textcolor{comment}{ *   @date   2016/11/24}
00004 \textcolor{comment}{ *   @author 佐伯雄飛}
00005 \textcolor{comment}{ *   @author B162392}
00006 \textcolor{comment}{ */}
00007 
00008 \textcolor{preprocessor}{#include <stdio.h>}
00009 \textcolor{preprocessor}{#include <stdbool.h>}  \textcolor{comment}{// bool}
00010 \textcolor{comment}{}
00011 \textcolor{comment}{/** @fn     void show\_items(char *items[], int n)}
00012 \textcolor{comment}{ *  @brief  メニュー項目を表示する}
00013 \textcolor{comment}{ *  @param  items メニュー項目}
00014 \textcolor{comment}{ *  @param  n 項目数}
00015 \textcolor{comment}{ *  @date   2016/06/08}
00016 \textcolor{comment}{ *  @author Toru Tamaki}
00017 \textcolor{comment}{ */}
00018 \textcolor{keywordtype}{void} show_items(\textcolor{keywordtype}{char}* items[], \textcolor{keywordtype}{int} n) \{
00019   \textcolor{keywordflow}{for} (\textcolor{keywordtype}{int} i = 1; i <= n; i++) \{
00020     printf(\textcolor{stringliteral}{"%s "}, items[i]);
00021   \}
00022   printf(\textcolor{stringliteral}{"\(\backslash\)n"});
00023 \}
00024 \textcolor{comment}{}
00025 \textcolor{comment}{/** @fn     show\_menu(char **menu[], int n)}
00026 \textcolor{comment}{ *  @brief  メニューを表示する}
00027 \textcolor{comment}{ *  @param  menu メニュー}
00028 \textcolor{comment}{ *  @param  n メニュー数}
00029 \textcolor{comment}{ *  @date   2016/06/08}
00030 \textcolor{comment}{ *  @author Toru Tamaki}
00031 \textcolor{comment}{ */}
00032 \textcolor{keywordtype}{void} show_menu(\textcolor{keywordtype}{char}** menu[], \textcolor{keywordtype}{int} n) \{
00033   \textcolor{keywordflow}{for} (\textcolor{keywordtype}{int} i = 0; i < n; i++) \{
00034     printf(\textcolor{stringliteral}{"%s "}, menu[i][0]);
00035   \}
00036   printf(\textcolor{stringliteral}{"\(\backslash\)n"});
00037 \}
00038 \textcolor{comment}{}
00039 \textcolor{comment}{/** @fn int main(void)}
00040 \textcolor{comment}{ *  @brief メニュー項目を表示する}
00041 \textcolor{comment}{ *}
00042 \textcolor{comment}{ *  入力:}
00043 \textcolor{comment}{ *  - 標準入力から,メニュー項目を表すアルファベット1文字が与えられる}
00044 \textcolor{comment}{ *  - もしくは終了(quit)を意味する'q'が与えられる}
00045 \textcolor{comment}{ *  - 文字が複数与えられる場合,空白(もしくは改行)で区切られている}
00046 \textcolor{comment}{ *}
00047 \textcolor{comment}{ *  出力:}
00048 \textcolor{comment}{ *  - メニューmenuを表示する}
00049 \textcolor{comment}{ *  -}
00050 \textcolor{comment}{そして入力されたアルファベット1文字に対応するメニュー項目の内容を表示する.}
00051 \textcolor{comment}{ *  - 文字'q'が与えられた場合,終了する.}
00052 \textcolor{comment}{ *  - それ以外の文字が入力された場合,menu表示に戻る}
00053 \textcolor{comment}{ *  -}
00054 \textcolor{comment}{空白(もしくは改行)の後に続けて文字が与えられた場合,上記の処理を繰り返す.}
00055 \textcolor{comment}{ *}
00056 \textcolor{comment}{ *  注意:}
00057 \textcolor{comment}{ *  - メニュー項目 File の場合,先頭の1文字 F が対応するアルファベットである.}
00058 \textcolor{comment}{ *  - 以下のコードでは4つのメニュー項目が定義されている.}
00059 \textcolor{comment}{\(\backslash\)verbatim}
00060 \textcolor{comment}{  char *file[] = \{"File(F)", "open",       "close",    "new",      "quit"\};}
00061 \textcolor{comment}{  char *edit[] = \{"Edit(E)", "undo",       "paste",    "copy",     "delete"\};}
00062 \textcolor{comment}{  char *view[] = \{"View(V)", "fullscreen", "minimize", "large",    "small"\};}
00063 \textcolor{comment}{  char *help[] = \{"Help(H)", "help",       "search",   "document", "website"\};}
00064 \textcolor{comment}{  char **menu[] = \{file, edit, view, help\};}
00065 \textcolor{comment}{\(\backslash\)endverbatim}
00066 \textcolor{comment}{ *}
00067 \textcolor{comment}{これら以外にもメニューを追加することを考慮して,入力文字がFかEかを判定するというコードを書かず,}
00068 \textcolor{comment}{ *    一般的なコードを書くこと.}
00069 \textcolor{comment}{ *}
00070 \textcolor{comment}{ *  実行例(キーボード入力の場合):}
00071 \textcolor{comment}{\(\backslash\)verbatim}
00072 \textcolor{comment}{$ ./main}
00073 \textcolor{comment}{File(F) Edit(E) View(V) Help(H)}
00074 \textcolor{comment}{F}
00075 \textcolor{comment}{open close new quit}
00076 \textcolor{comment}{File(F) Edit(E) View(V) Help(H)}
00077 \textcolor{comment}{E}
00078 \textcolor{comment}{undo paste copy delete}
00079 \textcolor{comment}{File(F) Edit(E) View(V) Help(H)}
00080 \textcolor{comment}{V}
00081 \textcolor{comment}{fullscreen minimize large small}
00082 \textcolor{comment}{File(F) Edit(E) View(V) Help(H)}
00083 \textcolor{comment}{H}
00084 \textcolor{comment}{help search document website}
00085 \textcolor{comment}{File(F) Edit(E) View(V) Help(H)}
00086 \textcolor{comment}{D}
00087 \textcolor{comment}{File(F) Edit(E) View(V) Help(H)}
00088 \textcolor{comment}{q}
00089 \textcolor{comment}{$}
00090 \textcolor{comment}{\(\backslash\)endverbatim}
00091 \textcolor{comment}{  *  標準入力例:}
00092 \textcolor{comment}{\(\backslash\)verbatim}
00093 \textcolor{comment}{F F E V}
00094 \textcolor{comment}{\(\backslash\)endverbatim}
00095 \textcolor{comment}{  *  出力例:}
00096 \textcolor{comment}{\(\backslash\)verbatim}
00097 \textcolor{comment}{File(F) Edit(E) View(V) Help(H)}
00098 \textcolor{comment}{open close new quit}
00099 \textcolor{comment}{File(F) Edit(E) View(V) Help(H)}
00100 \textcolor{comment}{open close new quit}
00101 \textcolor{comment}{File(F) Edit(E) View(V) Help(H)}
00102 \textcolor{comment}{undo paste copy delete}
00103 \textcolor{comment}{File(F) Edit(E) View(V) Help(H)}
00104 \textcolor{comment}{fullscreen minimize large small}
00105 \textcolor{comment}{File(F) Edit(E) View(V) Help(H)}
00106 \textcolor{comment}{\(\backslash\)endverbatim}
00107 \textcolor{comment}{ *  @date   2016/11/24}
00108 \textcolor{comment}{ *  @author 佐伯雄飛,B162392}
00109 \textcolor{comment}{ */}
00110 \textcolor{keywordtype}{int} main(\textcolor{keywordtype}{void}) \{
00111   \textcolor{keywordtype}{char}* file[] = \{\textcolor{stringliteral}{"File(F)"}, \textcolor{stringliteral}{"open"}, \textcolor{stringliteral}{"close"}, \textcolor{stringliteral}{"new"}, \textcolor{stringliteral}{"quit"}\};
00112   \textcolor{keywordtype}{char}* edit[] = \{\textcolor{stringliteral}{"Edit(E)"}, \textcolor{stringliteral}{"undo"}, \textcolor{stringliteral}{"paste"}, \textcolor{stringliteral}{"copy"}, \textcolor{stringliteral}{"delete"}\};
00113   \textcolor{keywordtype}{char}* view[] = \{\textcolor{stringliteral}{"View(V)"}, \textcolor{stringliteral}{"fullscreen"}, \textcolor{stringliteral}{"minimize"}, \textcolor{stringliteral}{"large"}, \textcolor{stringliteral}{"small"}\};
00114   \textcolor{keywordtype}{char}* help[] = \{\textcolor{stringliteral}{"Help(H)"}, \textcolor{stringliteral}{"help"}, \textcolor{stringliteral}{"search"}, \textcolor{stringliteral}{"document"}, \textcolor{stringliteral}{"website"}\};
00115   \textcolor{keywordtype}{char}** menu[] = \{file, edit, view, help\};
00116   \textcolor{keywordtype}{int} num\_menu = 4;
00117 
00118   \textcolor{keywordtype}{int} len = 100;
00119   \textcolor{keywordtype}{char} input[100];
00120   \textcolor{keywordtype}{bool} is\_end = \textcolor{keyword}{false};
00121 
00122   \textcolor{keywordflow}{while} (!is\_end) \{
00123     show_menu(menu, num\_menu);
00124 
00125     \textcolor{keywordtype}{int} return\_value = scanf(\textcolor{stringliteral}{"%99s"}, input);
00126 
00127     \textcolor{keywordflow}{if} (return\_value == 0 ||    \textcolor{comment}{// if input didn't match to %s}
00128         return\_value == EOF) \{  \textcolor{comment}{// if there are no input}
00129       \textcolor{keywordflow}{break};                    \textcolor{comment}{// break while}
00130     \}
00131 
00132     \textcolor{keywordflow}{if} (\textcolor{charliteral}{'q'} == input[0]) \{  \textcolor{comment}{// if input is 'q'}
00133       \textcolor{keywordflow}{break};                \textcolor{comment}{// break while}
00134     \}
00135     \textcolor{comment}{// printf("%c",*(*menu[0]+5));}
00136     \textcolor{keywordflow}{for} (\textcolor{keywordtype}{int} n = 0; n < num\_menu; n++) \{
00137       \textcolor{keywordflow}{if} (input[0] == *(*menu[n] + 5)) \{
00138         show_items(menu[n], 4);
00139       \}
00140 
00141       \textcolor{comment}{// 入力文字がどのメニュー項目に相当するのかを判定し,show\_items()で表示してください}
00142     \}
00143   \}
00144 
00145   \textcolor{keywordflow}{return} 0;
00146 \}
\end{DoxyCode}
