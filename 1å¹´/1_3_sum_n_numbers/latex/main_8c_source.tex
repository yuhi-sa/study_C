\subsection{main.\-c}

\begin{DoxyCode}
00001 \textcolor{comment}{/**  @file main.c}
00002 \textcolor{comment}{ *   @brief  n個の実数の和を表示する}
00003 \textcolor{comment}{ *   @date   2016/10/17}
00004 \textcolor{comment}{ *   @author 佐伯雄飛}
00005 \textcolor{comment}{ *   @author B162392}
00006 \textcolor{comment}{ */}
00007 
00008 \textcolor{preprocessor}{#include <stdio.h>}
00009 \textcolor{comment}{}
00010 \textcolor{comment}{/** @fn int main(void)}
00011 \textcolor{comment}{ *  @brief  n個の実数の和を表示する}
00012 \textcolor{comment}{ *}
00013 \textcolor{comment}{ *  入力:}
00014 \textcolor{comment}{ *  - 標準入力の最初は実数の個数n(正の整数)}
00015 \textcolor{comment}{ *  - それに引き続いてn個の実数が与えられる}
00016 \textcolor{comment}{ *}
00017 \textcolor{comment}{ *  出力:}
00018 \textcolor{comment}{ *  - n個の整数の和を(小数点第1位まで)表示する}
00019 \textcolor{comment}{ *}
00020 \textcolor{comment}{ *  入力例:}
00021 \textcolor{comment}{\(\backslash\)verbatim}
00022 \textcolor{comment}{10 0.17092690844 0.269405693204 0.272729853937 0.83502923788 0.539916669772}
00023 \textcolor{comment}{0.186507679728 0.660337164448 0.335226340127 0.440513411122 0.75002980058}
00024 \textcolor{comment}{\(\backslash\)endverbatim}
00025 \textcolor{comment}{  *  出力例:}
00026 \textcolor{comment}{\(\backslash\)verbatim}
00027 \textcolor{comment}{4.5}
00028 \textcolor{comment}{\(\backslash\)endverbatim}
00029 \textcolor{comment}{  *  入力例:}
00030 \textcolor{comment}{\(\backslash\)verbatim}
00031 \textcolor{comment}{6 0.344080271438 0.462451151037 0.76436728616 0.592085694456 0.145033025297}
00032 \textcolor{comment}{0.167567732804}
00033 \textcolor{comment}{\(\backslash\)endverbatim}
00034 \textcolor{comment}{  *  出力例:}
00035 \textcolor{comment}{\(\backslash\)verbatim}
00036 \textcolor{comment}{2.5}
00037 \textcolor{comment}{\(\backslash\)endverbatim}
00038 \textcolor{comment}{ *  @date   2016/10/17}
00039 \textcolor{comment}{ *  @author 佐伯雄飛,B162392}
00040 \textcolor{comment}{ */}
00041 \textcolor{keywordtype}{int} main(\textcolor{keywordtype}{void}) \{
00042   \textcolor{keywordtype}{int} n;
00043   scanf(\textcolor{stringliteral}{"%d"}, &n);
00044 
00045   \textcolor{keywordtype}{float} sum = 0;
00046   \textcolor{keywordflow}{for} (\textcolor{keywordtype}{int} i = 1; i <= n; i++) \{
00047     \textcolor{keywordtype}{float} number;
00048     scanf(\textcolor{stringliteral}{"%f"}, &number);
00049     sum += number;
00050   \}
00051 
00052   printf(\textcolor{stringliteral}{"%.1f\(\backslash\)n"}, sum);  \textcolor{comment}{// 小数点第1位まで表示させる}
00053 
00054   \textcolor{keywordflow}{return} 0;
00055 \}
\end{DoxyCode}
