\subsection{main.\-c}

\begin{DoxyCode}
00001 \textcolor{comment}{/**  @file main.c}
00002 \textcolor{comment}{ *   @brief  2個の実数に演算を適用する}
00003 \textcolor{comment}{ *   @date   2016/10/17}
00004 \textcolor{comment}{ *   @author 佐伯雄飛}
00005 \textcolor{comment}{ *   @author B162392}
00006 \textcolor{comment}{ */}
00007 
00008 \textcolor{preprocessor}{#include <stdio.h>}
00009 \textcolor{comment}{}
00010 \textcolor{comment}{/** @fn int main(void)}
00011 \textcolor{comment}{ *  @brief  2個の実数に演算を適用する}
00012 \textcolor{comment}{ *}
00013 \textcolor{comment}{ *  入力:}
00014 \textcolor{comment}{ *  - 標準入力の最初は演算子を表す文字(+, -, *, / のいずれか)}
00015 \textcolor{comment}{ *  - それに引き続く2個の実数は演算子を適用する数値}
00016 \textcolor{comment}{ *}
00017 \textcolor{comment}{ *  出力:}
00018 \textcolor{comment}{ *  - 2個の数値に演算子を適用した結果の数値を出力する}
00019 \textcolor{comment}{ *  - 数値は小数点第2位まで表示する(%.2f).}
00020 \textcolor{comment}{ *}
00021 \textcolor{comment}{ *  入力例:}
00022 \textcolor{comment}{\(\backslash\)verbatim}
00023 \textcolor{comment}{+ 0.912665355362 0.306135641218}
00024 \textcolor{comment}{\(\backslash\)endverbatim}
00025 \textcolor{comment}{  *  出力例:}
00026 \textcolor{comment}{\(\backslash\)verbatim}
00027 \textcolor{comment}{1.22}
00028 \textcolor{comment}{\(\backslash\)endverbatim}
00029 \textcolor{comment}{  *  入力例:}
00030 \textcolor{comment}{\(\backslash\)verbatim}
00031 \textcolor{comment}{/ 0.914028521038 0.759992013169}
00032 \textcolor{comment}{\(\backslash\)endverbatim}
00033 \textcolor{comment}{  *  出力例:}
00034 \textcolor{comment}{\(\backslash\)verbatim}
00035 \textcolor{comment}{1.20}
00036 \textcolor{comment}{\(\backslash\)endverbatim}
00037 \textcolor{comment}{ *  @date   2016/10/17}
00038 \textcolor{comment}{ *  @author 佐伯雄飛,B162392}
00039 \textcolor{comment}{ */}
00040 \textcolor{keywordtype}{int} main(\textcolor{keywordtype}{void}) \{
00041   \textcolor{keywordtype}{char} \textcolor{keyword}{operator};
00042   scanf(\textcolor{stringliteral}{"%c"}, &\textcolor{keyword}{operator});
00043 
00044   \textcolor{keywordtype}{float} operand1;
00045   \textcolor{keywordtype}{float} operand2;
00046   scanf(\textcolor{stringliteral}{"%f"}, &operand1);
00047   scanf(\textcolor{stringliteral}{"%f"}, &operand2);
00048 
00049   \textcolor{keywordtype}{float} result;
00050   \textcolor{keywordflow}{switch} (\textcolor{keyword}{operator}) \{
00051     \textcolor{keywordflow}{case} \textcolor{charliteral}{'+'}:
00052       result = operand1 + operand2;
00053       \textcolor{keywordflow}{break};
00054     \textcolor{keywordflow}{case} \textcolor{charliteral}{'-'}:
00055       result = operand1 - operand2;
00056       \textcolor{keywordflow}{break};
00057     \textcolor{keywordflow}{case} \textcolor{charliteral}{'*'}:
00058       result = operand1 * operand2;
00059       \textcolor{keywordflow}{break};
00060     \textcolor{keywordflow}{case} \textcolor{charliteral}{'/'}:
00061       result = operand1 / operand2;
00062       \textcolor{keywordflow}{break};
00063     \textcolor{keywordflow}{default}:  \textcolor{comment}{// unknown operator}
00064       result = 0;
00065   \}
00066   printf(\textcolor{stringliteral}{"%.2f\(\backslash\)n"}, result);
00067 
00068   \textcolor{keywordflow}{return} 0;
00069 \}
\end{DoxyCode}
