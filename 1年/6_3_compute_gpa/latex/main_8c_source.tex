\subsection{main.\-c}

\begin{DoxyCode}
00001 \textcolor{comment}{/**  @file main.c}
00002 \textcolor{comment}{ *   @brief  GPAの計算}
00003 \textcolor{comment}{ *   @date   2017/1/10}
00004 \textcolor{comment}{ *   @author 佐伯雄飛}
00005 \textcolor{comment}{ *   @author B162392}
00006 \textcolor{comment}{ */}
00007 
00008 \textcolor{preprocessor}{#include <stdio.h>}
00009 \textcolor{comment}{}
00010 \textcolor{comment}{/** @fn int main(int argc, char* argv[])}
00011 \textcolor{comment}{ *  @brief GPAの計算}
00012 \textcolor{comment}{ *}
00013 \textcolor{comment}{ *  入力:}
00014 \textcolor{comment}{ *  - 標準入力には,まず学生数が与えられる}
00015 \textcolor{comment}{ *  - 続いて講義数が与えられる.それに引き続いて,}
00016 \textcolor{comment}{ *    空白で区切られた講義コード(6桁)が講義数分与えられる.}
00017 \textcolor{comment}{ *  - 続いて「学生番号(6桁),講義コード,素点(100点満点)」が}
00018 \textcolor{comment}{ *    (空白で区切られて)与えられる.}
00019 \textcolor{comment}{ *    このフォーマットは,一人の学生について異なる講義コードが連続しており,}
00020 \textcolor{comment}{ *    その学生の成績がすべて与えられたら,別の学生の成績が与えられる.}
00021 \textcolor{comment}{ *  - 仮定:すべての学生は同じ講義を受講していると仮定する}
00022 \textcolor{comment}{ *    (入力はこの仮定を満たしている)}
00023 \textcolor{comment}{ *}
00024 \textcolor{comment}{ *  出力:}
00025 \textcolor{comment}{ *  - 各学生について,GPAを計算する.}
00026 \textcolor{comment}{ *  - ある講義の成績:点数が90点〜100点とき,秀.}
00027 \textcolor{comment}{ *                 点数が80点〜89点とき,優.}
00028 \textcolor{comment}{ *                 点数が70点〜79点とき,良.}
00029 \textcolor{comment}{ *                 点数が60点〜69点とき,可.}
00030 \textcolor{comment}{ *  - ある講義の点数が60点以上のとき,その講義に合格し,2単位を取得する.}
00031 \textcolor{comment}{ *  - GPA = (秀の単位数 * 4 + 優の単位数 * 3 + 良の単位数 * 2 + 可の単位数 * 1)}
00032 \textcolor{comment}{ *           / (総登録単位数 * 4)}
00033 \textcolor{comment}{ *           * 100}
00034 \textcolor{comment}{ *  - 出力は,学生番号(%06d),GPA(%.2f)の順に,空白で区切り,改行する.}
00035 \textcolor{comment}{ *}
00036 \textcolor{comment}{ *  入力例(途中省略):}
00037 \textcolor{comment}{\(\backslash\)verbatim}
00038 \textcolor{comment}{150}
00039 \textcolor{comment}{20}
00040 \textcolor{comment}{030323 044417 112271 166084 230334 288383 304507 343261 375762 408164 415311}
00041 \textcolor{comment}{417011 467643 486000 518187 618065 674215 712181 755421 883186}
00042 \textcolor{comment}{002487 030323 74}
00043 \textcolor{comment}{002487 044417 41}
00044 \textcolor{comment}{002487 112271 64}
00045 \textcolor{comment}{002487 166084 38}
00046 \textcolor{comment}{002487 230334 72}
00047 \textcolor{comment}{002487 288383 82}
00048 \textcolor{comment}{002487 304507 67}
00049 \textcolor{comment}{...}
00050 \textcolor{comment}{886870 417011 85}
00051 \textcolor{comment}{886870 467643 61}
00052 \textcolor{comment}{886870 486000 75}
00053 \textcolor{comment}{886870 518187 88}
00054 \textcolor{comment}{886870 618065 76}
00055 \textcolor{comment}{886870 674215 73}
00056 \textcolor{comment}{886870 712181 69}
00057 \textcolor{comment}{886870 755421 90}
00058 \textcolor{comment}{886870 883186 83}
00059 \textcolor{comment}{\(\backslash\)endverbatim}
00060 \textcolor{comment}{  *  出力例(途中省略):}
00061 \textcolor{comment}{\(\backslash\)verbatim}
00062 \textcolor{comment}{002487 48.75}
00063 \textcolor{comment}{023065 48.75}
00064 \textcolor{comment}{038034 42.50}
00065 \textcolor{comment}{038177 45.00}
00066 \textcolor{comment}{044436 57.50}
00067 \textcolor{comment}{050530 46.25}
00068 \textcolor{comment}{...}
00069 \textcolor{comment}{874131 62.50}
00070 \textcolor{comment}{875426 47.50}
00071 \textcolor{comment}{880624 43.75}
00072 \textcolor{comment}{881667 50.00}
00073 \textcolor{comment}{882464 57.50}
00074 \textcolor{comment}{886516 38.75}
00075 \textcolor{comment}{886870 56.25}
00076 \textcolor{comment}{\(\backslash\)endverbatim}
00077 \textcolor{comment}{ *  @date   2017/1/10}
00078 \textcolor{comment}{ *  @author 佐伯雄飛,B162392}
00079 \textcolor{comment}{ */}
00080 \textcolor{keywordtype}{int} main(\textcolor{keywordtype}{int} argc, \textcolor{keywordtype}{char}* argv[]) \{
00081   \textcolor{keywordtype}{int} num\_students;
00082   scanf(\textcolor{stringliteral}{"%d"}, &num\_students);
00083 
00084   \textcolor{keywordtype}{int} num\_class;
00085   scanf(\textcolor{stringliteral}{"%d"}, &num\_class);
00086 
00087   \textcolor{keywordtype}{int} class\_codes[num\_class];
00088   \textcolor{keywordflow}{for} (\textcolor{keywordtype}{int} i = 0; i < num\_class; i++) \{
00089     scanf(\textcolor{stringliteral}{"%d"}, &class\_codes[i]);
00090   \}
00091 
00092   \textcolor{keywordtype}{int} student\_id, class\_code, score;
00093 
00094   \textcolor{keywordflow}{for} (\textcolor{keywordtype}{int} s = 0; s < num\_students; s++) \{
00095     \textcolor{keywordtype}{float} a = 0;
00096     \textcolor{keywordtype}{float} b = 0;
00097     \textcolor{keywordflow}{for} (\textcolor{keywordtype}{int} c = 0; c < num\_class; c++) \{
00098       scanf(\textcolor{stringliteral}{"%d"}, &student\_id);
00099       scanf(\textcolor{stringliteral}{"%d"}, &class\_code);
00100       scanf(\textcolor{stringliteral}{"%d"}, &score);
00101       \textcolor{keywordflow}{if} (score >= 90)
00102         a += 4;
00103       \textcolor{keywordflow}{else} \textcolor{keywordflow}{if} (score >= 80)
00104         a += 3;
00105       \textcolor{keywordflow}{else} \textcolor{keywordflow}{if} (score >= 70)
00106         a += 2;
00107       \textcolor{keywordflow}{else} \textcolor{keywordflow}{if} (score >= 60)
00108         a += 1;
00109     \}
00110     \textcolor{keywordtype}{int} d = num\_class * 4;
00111     b = a / d * 100;
00112 
00113     printf(\textcolor{stringliteral}{"%06d %.2f\(\backslash\)n"}, student\_id, b);
00114   \}
00115 
00116   \textcolor{keywordflow}{return} 0;
00117 \}
\end{DoxyCode}
