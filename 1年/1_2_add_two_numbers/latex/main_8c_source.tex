\subsection{main.\-c}

\begin{DoxyCode}
00001 \textcolor{comment}{/**  @file main.c}
00002 \textcolor{comment}{ *   @brief  2つの整数の和を表示する}
00003 \textcolor{comment}{ *   @date   2016/10/13}
00004 \textcolor{comment}{ *   @author 佐伯雄飛}
00005 \textcolor{comment}{ *   @author B162392}
00006 \textcolor{comment}{ */}
00007 
00008 \textcolor{preprocessor}{#include <stdio.h>}
00009 \textcolor{comment}{}
00010 \textcolor{comment}{/** @fn int main(void)}
00011 \textcolor{comment}{ *  @breaf  2つの整数の和を表示する}
00012 \textcolor{comment}{ *}
00013 \textcolor{comment}{ *  入力:}
00014 \textcolor{comment}{ *  - 標準入力に2つの整数が与えられる}
00015 \textcolor{comment}{ *}
00016 \textcolor{comment}{ *  出力:}
00017 \textcolor{comment}{ *  - 2個の整数の和を表示する}
00018 \textcolor{comment}{ *}
00019 \textcolor{comment}{ *  入力例:}
00020 \textcolor{comment}{\(\backslash\)verbatim}
00021 \textcolor{comment}{1 2}
00022 \textcolor{comment}{\(\backslash\)endverbatim}
00023 \textcolor{comment}{  *  出力例:}
00024 \textcolor{comment}{\(\backslash\)verbatim}
00025 \textcolor{comment}{3}
00026 \textcolor{comment}{\(\backslash\)endverbatim}
00027 \textcolor{comment}{  *  入力例:}
00028 \textcolor{comment}{\(\backslash\)verbatim}
00029 \textcolor{comment}{4 8}
00030 \textcolor{comment}{\(\backslash\)endverbatim}
00031 \textcolor{comment}{  *  出力例:}
00032 \textcolor{comment}{\(\backslash\)verbatim}
00033 \textcolor{comment}{12}
00034 \textcolor{comment}{\(\backslash\)endverbatim}
00035 \textcolor{comment}{ *  @date   2016/10/13}
00036 \textcolor{comment}{ *  @author 佐伯雄飛,B162392}
00037 \textcolor{comment}{ */}
00038 \textcolor{keywordtype}{int} main(\textcolor{keywordtype}{void}) \{
00039 
00040   \textcolor{keywordtype}{int} number1, number2;
00041   scanf(\textcolor{stringliteral}{"%d"}, &number1);
00042   scanf(\textcolor{stringliteral}{"%d"}, &number2);
00043 
00044   \textcolor{keywordtype}{int} number3 = number1 + number2;
00045 
00046   printf(\textcolor{stringliteral}{"%d\(\backslash\)n"}, number3); \textcolor{comment}{// これをnumber3にする}
00047 
00048   \textcolor{keywordflow}{return} 0;
00049 \}
\end{DoxyCode}
