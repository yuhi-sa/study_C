\subsection{main.\-c}

\begin{DoxyCode}
00001 \textcolor{comment}{/**  @file main.c}
00002 \textcolor{comment}{ *   @brief  a > b > cが成立するかどうかを判定する}
00003 \textcolor{comment}{ *   @date   2016/10/26}
00004 \textcolor{comment}{ *   @author 佐伯雄飛}
00005 \textcolor{comment}{ *   @author B162392}
00006 \textcolor{comment}{ */}
00007 
00008 \textcolor{preprocessor}{#include <stdio.h>}
00009 \textcolor{comment}{}
00010 \textcolor{comment}{/** @fn int main(void)}
00011 \textcolor{comment}{ *  @brief  a > b > cが成立するかどうかを判定する}
00012 \textcolor{comment}{ *}
00013 \textcolor{comment}{ *  入力:}
00014 \textcolor{comment}{ *  - 標準入力には3個の実数a, b, cがこの順番で与えられる.}
00015 \textcolor{comment}{ *}
00016 \textcolor{comment}{ *  出力:}
00017 \textcolor{comment}{ *  - \(\backslash\)f$ a > b > c \(\backslash\)f$が成立すれば1を,成立しなければ0を表示する}
00018 \textcolor{comment}{ *}
00019 \textcolor{comment}{ *  @date   2016/10/26}
00020 \textcolor{comment}{ *  @author 佐伯雄飛,B162392}
00021 \textcolor{comment}{ */}
00022 \textcolor{keywordtype}{int} main(\textcolor{keywordtype}{void}) \{
00023   \textcolor{keywordtype}{float} a, b, c;
00024   scanf(\textcolor{stringliteral}{"%f"}, &a);
00025   scanf(\textcolor{stringliteral}{"%f"}, &b);
00026   scanf(\textcolor{stringliteral}{"%f"}, &c);
00027 
00028   \textcolor{keywordflow}{if} (a > b) \{
00029     \textcolor{keywordflow}{if} (b > c) \{
00030       printf(\textcolor{stringliteral}{"1\(\backslash\)n"});
00031     \} \textcolor{keywordflow}{else} \{
00032       printf(\textcolor{stringliteral}{"0\(\backslash\)n"});
00033     \}
00034   \} \textcolor{keywordflow}{else} \{
00035     printf(\textcolor{stringliteral}{"0\(\backslash\)n"});
00036   \}
00037   \textcolor{keywordflow}{return} 0;
00038 \}
\end{DoxyCode}
