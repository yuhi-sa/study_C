\subsection{main.\-c}

\begin{DoxyCode}
00001 \textcolor{comment}{/**  @file main.c}
00002 \textcolor{comment}{ *   @brief  2次元(または3次元)行列の行列式を計算する}
00003 \textcolor{comment}{ *   @date   2016/10/26}
00004 \textcolor{comment}{ *   @author 佐伯雄飛}
00005 \textcolor{comment}{ *   @author B162392}
00006 \textcolor{comment}{ */}
00007 
00008 \textcolor{preprocessor}{#include <stdio.h>}
00009 \textcolor{comment}{}
00010 \textcolor{comment}{/** @fn int main(void)}
00011 \textcolor{comment}{ *  @brief  2次元(または3次元)行列の行列式を計算する}
00012 \textcolor{comment}{ *}
00013 \textcolor{comment}{ *  入力:}
00014 \textcolor{comment}{ *  - 標準入力には,まず行列が2次元か3次元かを区別する整数が与えられる.}
00015 \textcolor{comment}{ *    2次元なら2,3次元なら3.}
00016 \textcolor{comment}{ *}
00017 \textcolor{comment}{もしそれ以外の数値が与えられたら,プログラムは何も表示せずに終了する(return}
00018 \textcolor{comment}{0で).}
00019 \textcolor{comment}{ *  - それに引き続いて,2次元行列の要素}
00020 \textcolor{comment}{ *    (\(\backslash\)f$ a\_\{11\},a\_\{12\},a\_\{21\},a\_\{22\} \(\backslash\)f$)}
00021 \textcolor{comment}{ *    または3次元行列の要素}
00022 \textcolor{comment}{ *    (\(\backslash\)f$ a\_\{11\},a\_\{12\},a\_\{13\},a\_\{21\},a\_\{22\},a\_\{23\},a\_\{31\},a\_\{32\},a\_\{33\}\(\backslash\)f$)}
00023 \textcolor{comment}{ *    がこの順番で与えられる.}
00024 \textcolor{comment}{ *    値は実数.}
00025 \textcolor{comment}{ *}
00026 \textcolor{comment}{ *  出力:}
00027 \textcolor{comment}{ *  - 与えられた行列の行列式を計算し,それを標準出力に表示する.}
00028 \textcolor{comment}{ *  - 数値は小数点第5位まで表示する(%.5f).}
00029 \textcolor{comment}{ *}
00030 \textcolor{comment}{ *  入力例:}
00031 \textcolor{comment}{\(\backslash\)verbatim}
00032 \textcolor{comment}{2}
00033 \textcolor{comment}{0.933942777791 0.402727471670}
00034 \textcolor{comment}{0.181429362962 0.264556070201}
00035 \textcolor{comment}{\(\backslash\)endverbatim}
00036 \textcolor{comment}{  *  出力例:}
00037 \textcolor{comment}{\(\backslash\)verbatim}
00038 \textcolor{comment}{0.17401}
00039 \textcolor{comment}{\(\backslash\)endverbatim}
00040 \textcolor{comment}{  *  入力例:}
00041 \textcolor{comment}{\(\backslash\)verbatim}
00042 \textcolor{comment}{3}
00043 \textcolor{comment}{0.638833848053 0.0150411981842 0.402224237664}
00044 \textcolor{comment}{0.911355582360 0.8907163041810 0.419912154151}
00045 \textcolor{comment}{0.383181998496 0.2191199720820 0.678895763639}
00046 \textcolor{comment}{\(\backslash\)endverbatim}
00047 \textcolor{comment}{  *  出力例:}
00048 \textcolor{comment}{\(\backslash\)verbatim}
00049 \textcolor{comment}{0.26368}
00050 \textcolor{comment}{\(\backslash\)endverbatim}
00051 \textcolor{comment}{ *  @date   2016/10/26}
00052 \textcolor{comment}{ *  @author 佐伯雄飛,B162392}
00053 \textcolor{comment}{ */}
00054 
00055 \textcolor{keywordtype}{int} main(\textcolor{keywordtype}{void}) \{
00056   \textcolor{keywordtype}{int} n;
00057   scanf(\textcolor{stringliteral}{"%d"}, &n);
00058   \textcolor{keywordflow}{if} (n == 2 || n == 3) \{
00059     \textcolor{keywordtype}{float} A[n][n];  \textcolor{comment}{// matrix A}
00060 
00061     \textcolor{keywordflow}{for} (\textcolor{keywordtype}{int} i = 0; i < n; i++) \{
00062       \textcolor{keywordflow}{for} (\textcolor{keywordtype}{int} j = 0; j < n; j++) \{
00063         scanf(\textcolor{stringliteral}{"%f"}, &A[i][j]);
00064       \}
00065     \}
00066 
00067     \textcolor{keywordtype}{float} det;
00068     \textcolor{keywordflow}{if} (n == 2) \{
00069       det = A[0][0] * A[1][1] - A[1][0] * A[0][1];  \textcolor{comment}{// a\_11 a\_22 - a\_21 a\_12}
00070     \} \textcolor{keywordflow}{else} \{
00071       \textcolor{comment}{// n == 3の場合のコードをここに書いてください}
00072       det = A[0][0] * A[1][1] * A[2][2] - A[0][2] * A[1][1] * A[2][0] -
00073             A[1][0] * A[0][1] * A[2][2] + A[1][0] * A[2][1] * A[0][2] +
00074             A[0][1] * A[1][2] * A[2][0] - A[0][0] * A[1][2] * A[2][1];
00075     \}
00076     printf(\textcolor{stringliteral}{"%.5f\(\backslash\)n"}, det);
00077   \} \textcolor{keywordflow}{else} \{
00078     printf(\textcolor{stringliteral}{""});
00079   \}
00080   \textcolor{keywordflow}{return} 0;
00081 \}
\end{DoxyCode}
